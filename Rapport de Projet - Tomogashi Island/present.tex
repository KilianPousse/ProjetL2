\documentclass{article}
\usepackage{graphicx}
\usepackage[table]{xcolor}
\usepackage[T1]{fontenc}
\usepackage[french]{babel}
\usepackage[utf8]{inputenc}
\usepackage{hyperref}

% ------ Variables ------ %
\newcommand{\Tache}{\cellcolor{blue!25}}

\hypersetup{
    colorlinks=true,
    linkcolor=black,
    pdfpagemode=FullScreen,
}

\begin{document}

\begin{titlepage}
    % ------ TITRE ------ %
    \title{
        % ----- IMAGE D'ENTÊTE ----- %
        \includegraphics[width=0.25\textwidth]{logolemansU.png}
        \hspace{150pt}
        \includegraphics[width=0.25\textwidth]{logo_IC2.png}
        \\[2cm]
        \color{blue} \textbf{Le Mans Université} \\
        \color{black} Licence Informatique \textit{$2^{e}$ année} \\
        Module 174UP02 Rapport de Projet \\
        \textbf{Tomogashi Island}
    }
    \author{Alban Vallee, Elona Pierre, Kilian Pousse  }
    \date{6 avril 2024}
    \maketitle
    
    \abstract{Dans le cadre de nos études, nous avons réalisé un jeu vidéo qui nous a permis de mettre en avant tout ce que nous avions appris depuis le début de notre licence en informatique.} \newline 
    
    Lien du projet : \url{https://github.com/KilianPousse/Tomogashi_Island/}

\end{titlepage}

   

% ----- Page de sommaire ----- %

\newpage
\begin{center}
\section*{Sommaire}
\tableofcontents
\end{center}

% ----- Notre contenu ----- %

\newpage
\section{Introduction} 


\subsection{Principe du jeu}

Notre projet porte sur un jeu à l'aspect mignon avec de petits personnages représentant des animaux, le tout dans un univers coloré. Le joueur incarne un lapin nommé Haricot qui arrive sur une île sans savoir pourquoi ni comment il a atterri ici. Il y fera la rencontre de plusieurs personnages qui l'aideront à répondre à ses questions.\\ 

Le jeu que nous avons imaginé est inspiré de plusieurs jeux comme Stardew~Valley, Tomagochi, Animal~Crossing~et Harvest Moon.\\
\newline
\includegraphics[height = 2cm]{AClogo.png}
\hspace{7pt}
\includegraphics[height = 2cm]{HMlogo.jpg}
\hspace{7pt}
\includegraphics[height = 2cm]{SVlogo.jpeg}

\subsection{Règles}

Le joueur doit effectuer différentes tâches afin de découvrir comment il a atterri ici. Il le découvre au fur et à mesure grâce à des livres disséminés partout sur la carte. Le joueur peut également développer son économie ce qui lui permettra d'acheter des cosmétiques et différents items au cours de la partie. Il y a plusieurs moyens de récolter de l'argent que ce soit avec la pêche ou l'agriculture puisque le joueur pourra cultiver des plantations ou pécher du poisson pour obtenir des collectibles revendables sur la place du marché.
\\




%\begin{figure}[h]
 %   \centering
  %  \includegraphics[height = 2cm]{capture/blébebe.png}
   % \includegraphics[height = 2cm]{capture/blégrand.png} 
    %\caption{Exemple de pousse de blé}
%\end{figure}
%\begin{figure}[h]
%    \centering
 %   \includegraphics{capture/tomateGraine.png}
  %  \includegraphics{capture/TomateMoyen.png}
   % \includegraphics{capture/toamteAdulte.png}
   % \includegraphics{capture/tomateFini.png}
   % \caption{Exemple de pousse de tomate}
   % \label{fig:enter-label}
%\end{figure}





\newpage
\subsection{Fonctionnalités prévues}

Nous avions prévu une liste de fonctionnalités différentes que voici : 
\begin{enumerate}
         \item Faire pousser et récolter des cultures
         \item La pousse des cultures en fonction du temps
         \item Pêcher
         \item Acheter et vendre des objets
         \item Une carte dynamique contenant des éléments animés
         \item Des musiques pouvant changer en fonction d'où se situe le joueur sur la carte
         \item Des PNJ avec qui interagir
         \item Un menu
         \item La possibilité de sauvegarder la partie
         \item Une page de paramètre
         \item Une barre de raccourcis d'items
         \item Un inventaire
         \item Différentes quêtes
         \item Un système jour et nuit en fonction du temps qui passe
         \item Des crédits dans les menus
         \item Un tutoriel pour chaque interactions possibles
         \item Des coffres de rangement
         \item Une boutique de cosmétique
         \item Acheter et équiper des cosmétiques
         \item Créer un filtre pluie et faire varier la météo
         \item Une barre d'énergie qui descend plus vite si le joueur fait des actions
         \item Téléportation au lit du joueur lorsque qu'il est trop tard\newline 
         
\end{enumerate}


Nous n'avons malheureusement pas pu implémenter les fonctionnalités 17 à 22 pour des raisons multiples. Pour les fonctionnalités 18 et 19 qui concernent les cosmétiques, il y avait un manque crucial de temps pour les réaliser puisque nous aurions dû refaire les sprites du personnage et toutes ses animations pour faire bouger les cosmétiques avec ses mouvements. En gérer la boutique rajoutée d'autant plus de temps nécessaire. D'autres fonctionnalités étaient prévues dans un second plan comme la 20 ou la 21 ce qui explique pourquoi elles ne sont pas encore disponibles dans le jeu.  

\newpage

\section{Organisation} %1p
%Diagramme de Gantt
%La partie « organisation » sert à décrire la manière dont les différentes tâches ont été définies
%et réparties entre les membres du projet. Vous pouvez également indiquer les outils que vous
%avez utilisés (dépôt git, slack, trello…). A la fin de cette partie, on doit savoir qui a fait quoi, comment vous avez organisé le travail collectif et quel planning vous avez suivi.%

L'organisation d'un projet a une importance primordiale puisqu'elle définit dès le départ le bon déroulement du projet. Elle se base sur les compétences et l'expérience de chacun pour pouvoir avancer dans de bonnes conditions. C'est en s'aidant des cours de \textit{Conduite de projets} que l'on a pu planifier le projet en respectant les différentes phases d'un cycle de vie d'un projet à l'aide d'outils comme un \textit{Diagramme de Gantt}.

\subsection{Diagramme de Gantt}

Au commencement du projet, nous nous sommes rassemblés pour établir un diagramme de Gantt précis avec toutes les fonctionnalités prévues(voir~annexe~\ref{fig:gantt_previsionnel}).

Cependant, pour une meilleure compréhension générale, nous avons établi un diagramme de Gantt simplifié énonçant seulement les tâches principales.

\begin{figure}[h]
\centering
\includegraphics[height = 6cm]{ganttSimple.png}
\caption{Diagramme de Gantt simplifié}
\label{fig:gantt_simplifie}
\end{figure}
\newpage
%\subsection{Répartition des tâches}



\subsection{Répartition des tâches}

Nous nous sommes alors repartit les taches dès le départ puis tout au long du projet pour toujours avancer de manières égales.

\begin{figure}[h]
\begin{tabular}{|l|c|c|c|}
Tâches & Alban & Elona & Kilian \\
\hline
Diagramme de Gantt &\Tache & \Tache & \\
Implémentation de Doxygène & & \Tache & \Tache \\
Sprites des Personnages & & \Tache& \\
Cartes / Collisions & \Tache & & \\
Mouvements du personnage & & & \Tache \\
Interface du Menu & & \Tache & \Tache\\
Histoire &\Tache & &\\
Système horaire & & & \Tache \\
Activité : Agriculture & & & \Tache \\
Activité : Pêche &\Tache & & \Tache \\
Activité: Marché & &\Tache & \Tache \\
Musiques &\Tache & \Tache &\Tache \\
Barre des tâches &\Tache & & \\
Gestion des items/inventaires & & &\Tache \\
Monnaie virtuelle & & &\Tache \\
Gestion des dialogues & & &\Tache \\
Options & &\Tache & \\
Filtre jour/nuit & & &\Tache \\
Tests & \Tache & \Tache &\Tache \\
Oral de soutenance & \Tache & \Tache &\Tache \\
Rédaction du \LaTeX &\Tache &\Tache &\Tache \\


\end{tabular}
\caption{Tableau présentant la répartition des tâches.}\label{f1}
\end{figure}





\subsection{Respect des deadlines} %Ajoue

Il a fallu repenser plusieurs fois notre organisation durant le projet afin de pouvoir arriver à la finalité de notre projet. Malheureusement, nous n'avons pas pu développer certaines fonctionnalités à cause du manque de temps. Nous avions prévu cette éventualité et avons donc réalisé le projet de manière à implémenter les objectifs les plus importants en priorité de sorte à ce que le projet puisse être fonctionnel tout en ayant une physionomie de jeu intéressante.

\subsection{Division des tâches}

Les taches du projet ont été divisées en plusieurs blocs pour prévoir le manque de temps possible pour l'implémentation des dernières fonctionnalités du jeu. Nous avons commencé par fabriquer la base du jeu avec les cartes, les sprites puis nous avons découpé chaque fonctionnalité en partie de code : le menu, les options, l'agriculture, la pêche, le marché, l'histoire, etc.




\section{Conception}
Nous avons décidé d'organiser nos fonctionnalités en différents blocs afin de pouvoir les développer séparément sans se marcher dessus. Cette méthode nous a permis d'avancer étape par étape sur les taches, de la plus importante à la plus optionnelle. Ce modèle de travail nous a octroyé la possibilité de rendre un jeu avec les fonctionnalités principales, comme les collisions, l'agriculture, etc. Le tout dans un temps définit. Nous avons pu également en ajouter certaines moins primaires comme le filtre nuit ou bien les dialogues avec les PNJs. 
    
\subsection{Conception du projet}

 Notre espace de travail est structuré comme vu durant le coura de conduite de projet. Les fichiers sont regroupés selon leur extensions et leur utilités. Par exemple, les fichiers \textit{*.h} se retrouvent dans le dossier \textit{include}. Cette arborescence permet une bonne organisation de notre travail. Chaque section, chaque fonctionnalité se trouve dans un fichier adéquat. Si nous prenons l'exemple des fonctions de gain et de suppression d'objets. Elles consistent à modifier le contenu de ce dernier. On a donc décidé de mettre les fonctions \textit{inventary\textunderscore give} et \textit{inventary\textunderscore remove} dans le fichier \textit{src/inventary.c}.

% ---- Arborescence ---- %
\begin{figure}[h]  
    \begin{itemize}
        \item assets
        \item bin
        \item include
        \begin{itemize}
            \item header.h \textit{(Entête principal pour tous les scriptes)}
            \item item.h \textit{(Liste des items)}
            \item style.h \textit{(Lien vers les images dans les dossiers)}
            \item [...]
        \end{itemize}
        \item src
        \begin{itemize}
            \item main.c \textit{(Programme principal}
            \item player.c  \textit{(Gestion, affichage du joueur)}
            \item display.c \textit{(Affichage des arrières plans, calques, ...)}
            \item map.c \textit{(Gestions des cartes)}
            \item farm.c \textit{(Gestion de l'agriculture)}
            \item [...]
        \end{itemize}
        \item test
    \end{itemize}
    \centering
    \caption{Arborescence}
    \label{fig:arborescence}
\end{figure}

\newpage
\subsection{Les sprites}


Tout nos sprites animés ont été réalisé par nous-mêmes : 
\begin{itemize}
    \item Les PNJ
    \item Les mouvements de l'eau
    \item Les objets décoratifs (le bateau, les arbres en mouvement,... )
    \item Le filtre pluie
    \item L'image du menu
    \item Le logo de notre jeu
    \item Le logo de la fenêtre
\end{itemize}
\subsubsection{Les personnages}


Pour réaliser ce genre de sprites de personnages il faut mettre bout à bout plusieurs images et les faire coïncider ensemble.

Tout nos sprites de personnages ont été créé sur la même base, c'est-à-dire quatre images par animation, le personnage principal est le seul à avoir des animations pour toutes les directions, les autres sont animés pour rester sur place et bouger de manière à les rendre plus vivant.
\begin{figure}[h]
    \centering
    \includegraphics[height = 2cm]{py/idle_0.png}
    \includegraphics[height = 2cm]{py/idle_1.png}
    \includegraphics[height = 2cm]{py/idle_2.png}
    \includegraphics[height = 2cm]{py/idle_3.png}
    \caption{Exemple de sprite d'un pnj volant}
\end{figure}

\begin{figure}[h]
    \centering
    \includegraphics[height = 2cm]{haricot/forward_0.png}
    \includegraphics[height = 2cm]{haricot/forward_1.png}
    \includegraphics[height = 2cm]{haricot/forward_2.png}
    \includegraphics[height = 2cm]{haricot/forward_3.png}
    \caption{Exemple de sprite de notre personnage principal}
\end{figure}

Les autres animations de notre personnage principal et PNJ sont retrouvable a la figure \ref{lapin_devant} de l'annexe.
\\
\newpage
\subsubsection{Les overlays}
Nous avons créé un système d'overlay afin de rendre la carte plus dynamique, cette couche de la carte affichera l'eau animée et les items de décorations animés comme le bateau. \newline

\begin{figure}[h]
    \centering
    \includegraphics[height = 1.5cm]{bateau/idle_0.png}
    \includegraphics[height = 1.5cm]{bateau/idle_1.png}
    \includegraphics[height = 1.5cm]{bateau/idle_2.png}
    \includegraphics[height = 1.5cm]{bateau/idle_3.png}
    \caption{Exemple d'overlay}
\end{figure}
Ce système permet de faire passer le personnage derrière ces éléments de décors. Il faut donc décomposer les cartes en quatre couches : les animations de l'eau (disponible en annexe sur 14 frames), les textures du sol et les ombres, les PNJ et enfin le personnage principal.

\begin{figure}[h]
    \centering
    \includegraphics[height = 5cm]{overlay.png}
    \caption{Exemple d'overlay d'une carte}
\end{figure}
\begin{figure}[h]
    \centering
    \includegraphics[height = 4cm]{background_0-1.png}
    \caption{Exemple de texture de sol}
\end{figure}
\newpage

\subsubsection{Les logo et image de fond}
Enfin, pour donner une identité visuelle à notre jeu nous avons crée nous-même le logo du jeu, l'icône de la fenêtre et les images de fond du menu. Nous avons également animé ces images à certains endroits dans notre jeu comme pour le logo dans les menus afin de rendre le jeu plus dynamique ainsi que certains éléments sur l'image de fond. Nous avons également réutilisé les personnages que nous avons créés et certains éléments reconnaissables qui définissent notre jeu à différents endroits comme dans le menu ou les options. %modif
%Ajouter des images
\begin{figure}[h]
    \centering
    \includegraphics[height = 4cm]{fond.png}
    \caption{Fond d'écran du menu}
\end{figure}
\begin{figure}[h]
    \centering
    \includegraphics[height = 3cm]{logo_1.png}
    \caption{Logo du jeu}
\end{figure}


\section{Développement}
    Le développement est la partie la plus conséquente de notre jeu projet. Ici, chaque fonctionnalité y sera détaillée. Chaque outil utilisé y sera aussi retranscrit afin de réaliser un développement global de cette partie.
\newpage
    \subsection{Logiciel et outils utilisés}
    % SDL %
    On a utilisé la seconde version de la librairie graphique Simple DirectMedia Layer (SDL) afin d’établir une interface graphique agréable et ergonomique au joueur. SDL apporte une aide considérable à la programmation d’un jeu utilisant le langage C. Il permet d’afficher des images, jouer des musiques, détecter des événements comme le déclenchement d’une touche. 

    % VSC %
    On a également utilisé Visual Studio Code comme outil de développement. Ce dernier nous a permis de programmer avec beaucoup d'aide et de raccourcis. 

    % Pixilart %
    
    \subsection{Détails des fonctionnalités}
        Comme dit précédemment dans la partie \textit{Conception du projet}, on a décidé de développer notre projet en bloque. Nous allons voir la méthode utilisée pour réaliser ces fonctionnalités.
\subsubsection{La carte}
La carte de Tomogashi Island utile un fonctionnement particulier que l’on peut retrouver dans différents jeux rétros. Le premier opus d’\textit{Animal Crossing}, \textit{Zelda 1} sont des jeux qui utilisent ce principe dans la présentation de leur carte. En effet, celle de Tomogashi Island est séparée en neuf parties distinctes qui ont une fonctionnalité toute différente. Nous pouvons y retrouver une zone d’agriculture, de pêche, de commerce, de plage, d'habitations. Ce choix s'explique par une volonté de rappeler l'époque de ces jeux pour apporter une touche de nostalgie.

        \begin{figure}[!h]  
            \includegraphics[height = 6cm]{découpage.png}
            \centering
            \caption{Exemple de collision}
            \label{fig:collions}
        \end{figure}

        Nous avons donc organisé les différentes parties des cartes dans un tableau se situant dans une structure \textit{map\textunderscore t}. Où pour chaque partie de cartes, on a une texture d'arrière-plan et de calque.
         
\subsubsection{Les collisions}
Chaque parcelle de la carte est découpée en un damier de tuiles. Chaque tuile possède une valeur qui représente une interaction avec le joueur. Si la valeur est négative, alors il ne pourra pas y aller. Cependant, la valeur peut indiquer qu'une action est possible par le joueur. Par exemple, les tuiles numérotées 33 signifient qu'à cet emplacement, il est possible de planter ou de récolter des ressources. Cette case est appelée « terre cultivable ».

La structure \textit{map\textunderscore t} contient les cartes à afficher ainsi que les valeurs des tuiles. Alors que \textit{player\textunderscore t} contient des informations sur le joueur comme sa position sur la carte. Cela permet de calculer la position des angles du spirite du personnage principal. Si le joueur souhait avancer alors qu'une des valeurs d'un des coins est négative, alors il y a collision et le joueur ne pourra pas avancer dans cette direction. (\textit{\hyperref[fig:collions]{Figure \ref{fig:collions}}} )

        \begin{figure}[!h]  
            \includegraphics[height = 4cm]{exermple_collision.png}
            \centering
            \caption{Exemple de collision}
            \label{fig:collions}
        \end{figure}
        
        \subsubsection{L'agriculture}
        Tomogashi Island a une économie basée sur la collecte de ressources agricoles. Le joueur doit acheter des graines pour les planter et les récolter afin de vendre le résultat.

        

    
Afin de gérer la croissance des plantations, on a utilisé une mécanique de cycle. Le jeu est réparti en journée de vingt-quarte heures. Chaque heure dure en réalité une minute. Cela signifie que si vous jouez durant vingt-quatre minutes, vous passez une journée dans le jeu. Afin de diminuer le temps d’attente entre la plantation et la récolte, on a décidé de sous-diviser une journée en quatre cycles de six heures. À la fin d’un cycle, toutes les plantations passent au stade de croissance supérieure. Quand une plante atteint son âge maximal, elle est enfin récoltable.

Un changement stade de croissance se remarque par un changement d'apparence. Par exemple, la tomate pousse en quatre stades qui se symbolise par une pousse qui grandit pour donner un pied de tomate.

        \begin{figure}[h]
            \centering
            \includegraphics[height = 2cm]{capture/blébebe.png}
            \includegraphics[height = 2cm]{capture/blégrand.png} 
            \caption{Exemple de pousse de blé}
        \end{figure}
        \begin{figure}[h]
            \centering
            \includegraphics{capture/tomateGraine.png}
            \includegraphics{capture/TomateMoyen.png}
            \includegraphics{capture/toamteAdulte.png}
            \includegraphics{capture/tomateFini.png}
            \caption{Exemple de pousse de tomate}
            \label{fig:enter-label}
        \end{figure}

        \newpage
        
        \subsubsection{La pêche}
        La pêche est un autre moyen pour de gagner de l'argent. La zone pêche se situe à gauche de la maison du joueur. Si le joueur se positionne au bout du ponton, il lui sera proposé de pouvoir pêcher. En appuyant sur une touche d'interaction, « espace » ou « e », le jeu se mettra en phase de pêche.

        \begin{figure}[h]
            \centering
            \includegraphics[height = 3cm]{peche_action.png}
            \includegraphics[height = 3cm]{peche.png}
            \includegraphics[height = 3cm]{peche_plop.png}
            \caption{Caption}
            \label{fig:enter-label}
        \end{figure}
        
        Après l'animation du personnage, un délai est choisi aléatoirement pouvant vaciller de trois à quinze secondes. Ce délai correspond au temps à attendre entre la fin du lancer et le plongeon du bouchon. Quand est submergé, le joueur à une seconde pour appuyer sur une touche d'interaction afin de remonter la ligne et de récupérer le poisson. Si le joueur n'appuie pas, alors un prochain délai sera calculer. Si le joueur réagis au bon moment, alors un poisson lui sera donné aléatoirement parmi cinq poissons plus ou moins rares. 
        
        \begin{figure} 
        \begin{tabular}{|l|c|}
            Race de Poisson & Taux d'obtention \\
            \hline
            Poisson          &  40\% \\
            Thon             &  30\% \\
            Poisson Lanterne &  20\% \\
            Carpe Koï        &   7\% \\
            Requin           &   3\% \\
        \end{tabular}
            
            \centering
            \caption{Exemple de collision}
            \label{fig:poissons}
        \end{figure}

        
        \subsubsection{L'inventaire}

        Il y a d'abord une barre de tâche toujours visible en jeu par l'utilisateur qui sert de mini inventaire avec quelques places disponible pour que le joueur puisse changer rapidement les objets qu'il tient en main.

        \begin{figure}[h]  
            \includegraphics[height = 2cm]{hotbar.png}
            \centering
            \caption{Barre des tâches}
            \label{fig:hotbar}
        \end{figure}

        
        
        L'inventaire est une fonctionnalité très importante dans Tomogashi Island. Cela permet de stocker des éléments appelés « \textit{item} ». Cet objet possède sa propre structure. Elle est composée de deux entiers: un pour connaître l'identifiant et l'autre le nombre d'un item.
        L'inventaire a également une structure constituée de deux tableaux d'\textit{items}, l'inventaire principal et la barre des tâches, ainsi qu'une variable indiquant l'objet sélectionne par le joueur.

        La barre des tâches permet d'accéder plus rapidement aux items afin que le joueur puisse les utiliser en les sélectionnant. 

        Le joueur peut toujours avoir accès au reste de l'inventaire en appuyant sur la touche « I ». Il pourra échanger les objets de place comme il le souhaite, afin d'utiliser celui de son choix.

        \begin{figure}[!h]  
            \includegraphics[height = 6cm]{inventory.png}
            \centering
            \caption{Inventaire}
            \label{fig:inventory}
        \end{figure}

       L'inventaire utilise des fonctions lui permettant l'ajout et la suppression d'items avec respectivement \textit{inventory\textunderscore give(item\textunderscore t * item)} et \\ \textit{inventory\textunderscore remove(item\textunderscore t *~item)}. 
        
Le fonctionnement global et la représentation de l'inventaire sont une inspiration du jeu vidéo \textit{Minecraft}.

        \newpage
        
        \subsubsection{Achat et vente d'objets}
L'économie est l'un des objectifs les plus important que propose le jeu. L'argent permet d'acheter des objets pouvant être utilisés dans l'agriculture par exemple. L'achat et la vente d'objets se font auprès des villageois se situant sur la place du marché de la zone commerçante. À l'aide d'une interface, le joueur peut choisir la quantité d'objets qui souhaite vendre ou bien acheter (\textit{\hyperref[fig:trade]{Figure\ref{fig:trade}}}).
        
Il existe différents types de marchand. Il y a le vendeur de graine, l'acheteur de plants, l'acheteur de poisson. À l'origine, nous avions pour objectif de proposer une boutique de cosmétique qui servirait au joueur pour dépenser son argent virtuel dans des accessoires qui seront portés par leur personnage.
\newpage
\subsection{Logiciel}
Durant le développement d'un projet, il est normal d'utiliser différents logiciels, lors de ce projet nous avons utilisé \textit{Pixilart} ce logiciel nous a permis de dessiner les cartes et de les séparer en différentes couches.
    
    \begin{figure}[h]  
        \includegraphics[height = 2cm]{pixilart.png}
        \centering
        \caption{Logo pixilart}
        \label{fig:logo1}
    \end{figure}


    L'utilisation de \textit{Piskel} nous a permis d'animer les sprites du personnage principal et des PNJ.

\begin{figure}[h]  
\includegraphics[height = 2cm]{piskel.jpeg}
\centering
\caption{Barre des tâches}
\label{fig:logo2}
\end{figure}



Le site internet \textit{Itch.io} a permis de gagner du temps sur certains éléments graphiques comme les objets de décorations type arbre, fleurs, ou encore champs et collectible (graine, légumes,etc.). 
    
    \begin{figure}[h]  
        \includegraphics[height = 2cm]{itch.io.png}
        \centering
        \caption{Barre des tâches}
        \label{fig:logo3}
    \end{figure}

\subsection{Les bibliothèques utilisée}
On a utilise la seconde version de la librairie graphique Simple DirectMedia Layer (SDL) comme énoncée plus haut pour l'interface graphique.

\newpage

\section{Tests et débogage}

\subsection{Tests}

Le jeu de test du projet est réparti en 3
\subsection{Débogage}




\section{Conclusion et résultats}
\subsection{Les difficultés rencontrées}

Ce projet aussi passionnant que chronophage nous a permis de gagner de l'expérience que ce soit en termes d'organisation, de communication, et d'apprentissage (SDL, Git, Doxygene, CUnit). Il nous a montré à quel point le partage des tâches est important dans un travail de groupe plus long que ceux que l'on fait habituellement.
Pour notre jeu, la majorité des implémentations voulues ont été ajoutées, pour celles qui ne le sont pas, il n'y avait pas de difficultés particulières rencontrées, il s'agissait surtout d'un manque de temps. Ce fut assez prévisible dès le départ, c'est pour cela que nous avons émis la possibilité de beaucoup de tâches pour être sûr de ne pas repasser par la phase de planification.
L'organisation via des outils comme le diagramme de Gantt n'a pas totalement fonctionné. S'adapter à ce format en étant en période de cours avec des horaires variables et du temps de travail personnel plus ou moins conséquents n'était pas une tâche facile. Ce n'est pas pour autant que le partage des tâches ne s'est pas fait sans soucis entre notre groupe, simplement pas grâce à un Trello ou à un Gantt.

Nous avons aussi dû apprendre à utiliser des outils comme SDL ou CUnit que nous ne connaissions pas. 

\subsection{Notre expérience}

Réaliser ce projet nous a permis d'utiliser des compétences acquises tout au long de nos études en informatique, ce projet nous a également permis d'apprendre à gérer le temps d'une manière plus optimale afin de respecter des délais.
\newpage
\section{Annexes}
 \subsection{Organisation}
    \begin{figure}[h]
    \centering
    \includegraphics[height = 7cm]{ganttEntier.png}
    \caption{Diagramme de Gantt prévisionnel}
    \label{fig:gantt_previsionnel}
    \end{figure}
    
\subsection{Les sprites}
\begin{figure}[h]
    \centering
    \includegraphics[height = 2cm]{risette/idle_0.png}
    \includegraphics[height = 2cm]{risette/idle_1.png}
    \includegraphics[height = 2cm]{risette/idle_2.png}
    \includegraphics[height = 2cm]{risette/idle_3.png}
    \\
    \includegraphics[height = 2cm]{brocoli/idle_0.png}
    \includegraphics[height = 2cm]{brocoli/idle_1.png}
    \includegraphics[height = 2cm]{brocoli/idle_2.png}
    \includegraphics[height = 2cm]{brocoli/idle_3.png}
    \\
    \includegraphics[height = 2cm]{ken/idle_0.png}
    \includegraphics[height = 2cm]{ken/idle_1.png}
    \includegraphics[height = 2cm]{ken/idle_2.png}
    \includegraphics[height = 2cm]{ken/idle_3.png}
    \\
    \includegraphics[height = 2cm]{serge/idle_0.png}
    \includegraphics[height = 2cm]{serge/idle_1.png}
    \includegraphics[height = 2cm]{serge/idle_2.png}
    \includegraphics[height = 2cm]{serge/idle_3.png}
    \\
    \includegraphics[height = 2cm]{zebi/idle_0.png}
    \includegraphics[height = 2cm]{zebi/idle_1.png}
    \includegraphics[height = 2cm]{zebi/idle_2.png}
    \includegraphics[height = 2cm]{zebi/idle_3.png}
    \caption{Les sprites PNJ}\label{risette}

    \includegraphics[height = 2cm]{haricot/idle_0.png}
    \includegraphics[height = 2cm]{haricot/idle_1.png}
    \includegraphics[height = 2cm]{haricot/idle_2.png}
    \includegraphics[height = 2cm]{haricot/idle_3.png}
    \\
    \includegraphics[height = 2cm]{haricot/back_0.png}
    \includegraphics[height = 2cm]{haricot/back_1.png}
    \includegraphics[height = 2cm]{haricot/back_2.png}
    \includegraphics[height = 2cm]{haricot/back_3.png}
    \\
    \includegraphics[height = 2cm]{haricot/left_0.png}
    \includegraphics[height = 2cm]{haricot/left_1.png}
    \includegraphics[height = 2cm]{haricot/left_2.png}
    \includegraphics[height = 2cm]{haricot/left_3.png}
    \caption{Exemple de sprite du personnage principal}\label{lapin_devant}
    \end{figure}

   
    \begin{figure}[h]
        \includegraphics[height = 6cm]{trade.png}
        \centering
        \caption{Exemple d'échange}
        \label{fig:trade}
    \end{figure}
        




\end{document}