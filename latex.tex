\documentclass{article}
\usepackage{graphicx}
\usepackage{xcolor}
\usepackage[T1]{fontenc}
\usepackage[french]{babel}
\usepackage[utf8]{inputenc}
\usepackage{hyperref}

\hypersetup{
    colorlinks=true,
    linkcolor=black,
    pdfpagemode=FullScreen,
}

\begin{document}

\begin{titlepage}
    % ------ TITRE ------ %
    \title{
        % ----- IMAGE D'ENTÊTE ----- %
        \includegraphics[width=0.25\textwidth]{logolemansU.png}
        \hspace{150pt}
        \includegraphics[width=0.25\textwidth]{logo_IC2.png}
        \\[2cm]
        \color{blue} \textbf{Le Mans Université} \\
        \color{black} Licence Informatique \textit{$2^{e}$ année} \\
        Module 174UP02 Rapport de Projet \\
        \textbf{Tomogashi Island}
    }
    \author{Alban, Elona, Kilian }
    \date{6 avril 2024}
    \maketitle
    
    \abstract{Dans le cadre de nos études nous avons réalisé un jeu vidéo qui nous a permis de mettre en avant tout ce que nous avions appris depuis le début de notre licence en informatique.}


\end{titlepage}

% ----- Page de sommaire ----- %
\newpage
\tableofcontents

% ----- Votre contenu ----- %
\newpage

\section{Introduction}
    \subsection{Principe du jeu}
    \subsection{Règle}
\section{Organisation}
\section{Conception}
\section{Developpement}
    \subsection{Logiciels}
    \subsection{Méthodes}
    \subsection{SDL}
\section{Conclusion et Résultat}



\end{document}
