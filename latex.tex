\documentclass{article}
\usepackage{graphicx}
\usepackage{xcolor}
\usepackage[T1]{fontenc}
\usepackage[french]{babel}
\usepackage[utf8]{inputenc}
\usepackage{hyperref}

\hypersetup{
    colorlinks=true,
    linkcolor=black,
    pdfpagemode=FullScreen,
}

\begin{document}

\begin{titlepage}
    % ------ TITRE ------ %
    \title{
        % ----- IMAGE D'ENTÊTE ----- %
        \includegraphics[width=0.25\textwidth]{logolemansU.png}
        \hspace{150pt}
        \includegraphics[width=0.25\textwidth]{logo_IC2.png}
        \\[2cm]
        \color{blue} \textbf{Le Mans Université} \\
        \color{black} Licence Informatique \textit{$2^{e}$ année} \\
        Module 174UP02 Rapport de Projet \\
        \textbf{Tomogashi Island}
    }
    \author{Alban, Elona, Kilian }
    \date{6 avril 2024}
    \maketitle
    
    \abstract{Dans le cadre de nos études nous avons réalisé un jeu vidéo qui nous a permis de mettre en avant tout ce que nous avions appris depuis le début de notre licence en informatique.}
\end{titlepage

% ----- Page de sommaire ----- %

\newpage
\begin{center}
\section*{Sommaire}
\tableofcontents
\end{center}

% ----- Notre contenu ----- %

\newpage
\section{Introduction}
\subsection{Principe du jeu}

Le jeu que nous avons imaginé est inspiré de Stardew Valley, Tomagochi, Animal Crossing et Harvest Moon. 
\\
\includegraphics[height = 2cm]{AClogo.png}
\hspace{50pt}
\includegraphics[height = 2cm]{HMlogo.jpg}
\hspace{50pt}
\includegraphics[height = 2cm]{SVlogo.jpeg} 
\\
C'est un jeu d'un aspect mignon avec de petits personnages representant des animaux, le tout dans un univers coloré. Le joueur incarne un petit personnage, Haricot, qui arrive sur une île sans savoir pourquoi ni comment il a atterri ici. Il y fera la rencontre de plusieurs personnages qui l'aideront dans sa quête : Découvrir pourquoi et comment est-il arrivé ici? 
\\

\subsection{Règles}

Le joueur doit effectuer plusieurs actions afin d'augmenter son niveau et passer les niveaux pour découvrir comment est-il arrivé ici. Il peut cultiver différentes plantations, avec un système de temps qui passe dans le jeu, il peut aussi pêcher dans un coin de pêche spécifique sur la map. Le joueur à aussi les moyens de vendre tout ce qu'il a pu récolter dans un marché sur une autre partie de la map.%Ajouter des images.
\subsection{Fonctionnalités prévues}

Nous avions prévu tout ensemble de fonctionnalités suivantes : 
\begin{enumerate}
         \item Faire pousser et récolter des cultures
         \item Vendre et acheter des objets
         \item Créer un filtre pluie et faire varier la météo
         \item Musique changeante en fonction d'où se situe le joueur dans la map
         \item Le temps qui passe dans le jeu, si il est trop tard le joueur est téléporté à son lit
         \item Des PNJ avec qui interagir
         \item Acheter et équiper des cosmétiques
         \item Une map dynamique avec des éléments animés
         \item Une boutique de cosmétique
         \item Un menu
         \item La possibilité de sauvegarder la partie
         \item Une page de paramètre
         \item Une barre d'énergie
         \item Des coffres de rangement
         \item Une barre d'item
         \item La pousse des cultures qui avance en fonction du temps
         \item Différentes quêtes
         \item Un tutoriel pour chaque interactions possibles
         \item Des crédits dans les menus
         \item La barre d'énergie descend plus vite si le joueur fait des actions
         \item Système jour et nuit en fonction du temps qui passe
         \item Pêcher
\end{enumerate}
Nous n'avons pas pu implémenter les fonctionnalités 7,9,13,20 et 22. nous n'avons pas eu le temps pour les faire, les fonctionnalités 7 et 9 auraient été trop compliquées à implémenter avec le temps imparti, nous aurions du refaire les sprites des personnages et toutes ses animations pour faire bouger les cosmétiques avec ses mouvements, la gestion de la boutique aurait été également trop compliqué à réaliser dans le temps imparti. Les fonctionnalités 13 et 22 étaient trop compliquée à réaliser nous ne savions pas comment nous y prendre. Et enfin la fonctionnalité 22 nous aurait demandé trop de temps.

\newpage
\section{Organisation}
\subsection{Diagramme de Gantt}
\subsection{Respect des deadlines}
\subsection{Division des tâches}
\section{Conception}
\subsection{Conception du projet}
\subsection{Les sprites}
\section{Développement}
\subsection{Logiciel}
\subsection{Méthodes}
\subsection{Les bibliothèques utilisée}
\section{Conception et résultats}
\subsection{Les difficultés rencontrées}
\subsection{Notre expèrience}
\section{Annexes}



\end{document}
