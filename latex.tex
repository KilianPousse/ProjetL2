\documentclass{article}
\usepackage{graphicx}
\usepackage{xcolor}
\usepackage[T1]{fontenc}
\usepackage[french]{babel}
\usepackage[utf8]{inputenc}
\usepackage{hyperref}

% ------ Variables ------ %
\newcommand{\Tache}{\cellcolor{blue!25}}

\hypersetup{
    colorlinks=true,
    linkcolor=black,
    pdfpagemode=FullScreen,
}

\begin{document}

\begin{titlepage}
    % ------ TITRE ------ %
    \title{
        % ----- IMAGE D'ENTÊTE ----- %
        \includegraphics[width=0.25\textwidth]{logolemansU.png}
        \hspace{150pt}
        \includegraphics[width=0.25\textwidth]{logo_IC2.png}
        \\[2cm]
        \color{blue} \textbf{Le Mans Université} \\
        \color{black} Licence Informatique \textit{$2^{e}$ année} \\
        Module 174UP02 Rapport de Projet \\
        \textbf{Tomogashi Island}
    }
    \author{Alban, Elona, Kilian }
    \date{6 avril 2024}
    \maketitle
    
    \abstract{Dans le cadre de nos études nous avons réalisé un jeu vidéo qui nous a permis de mettre en avant tout ce que nous avions appris depuis le début de notre licence en informatique.}
\end{titlepage}

% ----- Page de sommaire ----- %

\newpage
\begin{center}
\section*{Sommaire}
\tableofcontents
\end{center}

% ----- Notre contenu ----- %

\newpage
\section{Introduction}
\subsection{Principe du jeu}

Le jeu que nous avons imaginé est inspiré de Stardew Valley, Tomagochi, Animal Crossing et Harvest Moon. 
\\
\includegraphics[height = 2cm]{AClogo.png}
\hspace{50pt}
\includegraphics[height = 2cm]{HMlogo.jpg}
\hspace{50pt}
\includegraphics[height = 2cm]{SVlogo.jpeg} 
\\
C'est un jeu d'un aspect mignon avec de petits personnages representant des animaux, le tout dans un univers coloré. Le joueur incarne un petit personnage, Haricot, qui arrive sur une île sans savoir pourquoi ni comment il a atterri ici. Il y fera la rencontre de plusieurs personnages qui l'aideront dans sa quête : Découvrir pourquoi et comment est-il arrivé ici? 
\\

\subsection{Règles}

Le joueur doit effectuer plusieurs actions afin d'augmenter son niveau et passer les niveaux pour découvrir comment est-il arrivé ici. Il peut cultiver différentes plantations, avec un système de temps qui passe dans le jeu, il peut aussi pêcher dans un coin de pêche spécifique sur la map. Le joueur à aussi les moyens de vendre tout ce qu'il a pu récolter dans un marché sur une autre partie de la map.%Ajouter des images.
\subsection{Fonctionnalités prévues}

Nous avions prévu tout ensemble de fonctionnalités suivantes : 
\begin{enumerate}
         \item Faire pousser et récolter des cultures
         \item Vendre et acheter des objets
         \item Créer un filtre pluie et faire varier la météo
         \item Musique changeante en fonction d'où se situe le joueur dans la map
         \item Le temps qui passe dans le jeu, si il est trop tard le joueur est téléporté à son lit
         \item Des PNJ avec qui interagir
         \item Acheter et équiper des cosmétiques
         \item Une map dynamique avec des éléments animés
         \item Une boutique de cosmétique
         \item Un menu
         \item La possibilité de sauvegarder la partie
         \item Une page de paramètre
         \item Une barre d'énergie
         \item Des coffres de rangement
         \item Une barre d'item
         \item La pousse des cultures qui avance en fonction du temps
         \item Différentes quêtes
         \item Un tutoriel pour chaque interactions possibles
         \item Des crédits dans les menus
         \item La barre d'énergie descend plus vite si le joueur fait des actions
         \item Système jour et nuit en fonction du temps qui passe
         \item Pêcher
\end{enumerate}
Nous n'avons pas pu implémenter les fonctionnalités 7,9,13,20 et 22. nous n'avons pas eu le temps pour les faire, les fonctionnalités 7 et 9 auraient été trop compliquées à implémenter avec le temps imparti, nous aurions du refaire les sprites des personnages et toutes ses animations pour faire bouger les cosmétiques avec ses mouvements, la gestion de la boutique aurait été également trop compliqué à réaliser dans le temps imparti. Les fonctionnalités 13 et 22 étaient trop compliquée à réaliser nous ne savions pas comment nous y prendre. Et enfin la fonctionnalité 22 nous aurait demandé trop de temps.

\newpage


\section{Organisation} %1p
%Diagramme de Gantt
%La partie « organisation » sert à décrire la manière dont les différentes tâches ont été définies
%et réparties entre les membres du projet. Vous pouvez également indiquer les outils que vous
%avez utilisés (dépôt git, slack, trello…). A la fin de cette partie, on doit savoir qui a fait quoi, comment vous avez organisé le travail collectif et quel planning vous avez suivi.%

L'organisation d'un projet a une importance primordiale puisqu'elle définit dès le départ le bon déroulement du projet. Elle se base sur les compétences et l'expérience de chacun pour pouvoir avancer dans de bonnes conditions. C'est en s'aidant des cours de \textit{Conduite de projets} que l'on a pu planifier le projet en respectant les différentes phases d'un cycle de vie d'un projet à l'aide d'outils comme un \textit{Diagramme de Gantt}.

\subsection{Diagramme de Gantt}


\subsection{Répartition des tâches}

\begin{figure}[h]
\begin{tabular}{|l|c|c|c|}
Tâches & Alban & Elona & Kilian \\
\hline
Diagramme de Gantt &\Tache  &   \Tache  &     \\
Implémentation de Doxygène &    &   &    \Tache  \\
Sprites des Personnages &    &  \Tache&     \\
Cartes / Collisions & \Tache   &    &     \\%AF
Mouvements du personnage &    &  &  \Tache    \\
Interface du Menu &    &    \Tache  & \Tache\\%AF
Histoire &\Tache    &     &\\
Système horaire &    &    &   \Tache    \\
Activité : Agriculture &    &     &    \Tache    \\
Activité : Pêche &\Tache &  &  \Tache    \\%AF
Activité: Marché &    &\Tache &  \Tache    \\%AF
Musiques &\Tache    & \Tache  &\Tache  \\%AF
Barre des tâches &\Tache  &     &     \\
Gestion des items/inventaires &    &  &\Tache    \\
Monnaie virtuelle &    &  &\Tache    \\
Gestion des dialogues &    &  &\Tache    \\
Options &    &\Tache  &   \\%AF
Filtre pluie &\Tache    &\Tache  &\Tache  \\
Filtre jour/nuit &\Tache    &\Tache  &\Tache  \\
Tests & \Tache   & \Tache &\Tache    \\
Oral de soutenance & \Tache   & \Tache &\Tache    \\ %AF
Rédaction du \LaTeX &\Tache    &\Tache  &\Tache    \\%AF


\end{tabular}
\caption{Ceci est un tableau présentant la répartition des tâches.}\label{f1}
\end{figure}

Nouvelle ref \ref{f1}



\subsection{Respect des deadlines}
\subsection{Division des tâches}
\section{Conception}
\subsection{Conception du projet}
\subsection{Les sprites}
\subsection{Les sprites}
Tout nos sprites animés ont été réalisé par notre équipe : 
\begin{itemize}
    \item Les PNJ
    \item Les mouvements de l'eau
    \item Les objets décoratifs (le bateau, les arbres en mouvement,... )
    \item Le filtre pluie
    \item L'image du menu
    \item Le logo de notre jeu
    \item Le logo de la fenêtre
\end{itemize}
\subsubsection{Les personnages}
\begin{figure}[h]
    \centering
    \includegraphics[height = 2cm]{py/idle_0.png}
    \includegraphics[height = 2cm]{py/idle_1.png}
    \includegraphics[height = 2cm]{py/idle_2.png}
    \includegraphics[height = 2cm]{py/idle_3.png}
    \caption{Exemple de sprite d'un pnj volant}
\end{figure}

\begin{figure}[h]
    \centering
    \includegraphics[height = 2cm]{haricot/forward_0.png}
    \includegraphics[height = 2cm]{haricot/forward_1.png}
    \includegraphics[height = 2cm]{haricot/forward_2.png}
    \includegraphics[height = 2cm]{haricot/forward_3.png}
    \caption{Exemple de sprite de notre personnage principal}
\end{figure}
Les autres animations de notre personnage principal et PNJ sont retrouvable a la figure \ref{lapin_devant} de l'annexe.
\\
Pour réaliser ce genre de sprites il faut mettre bout à bout plusieurs images et les faire coïncider ensemble.\\
Tout nos sprites de personnages ont été crée sur la même base, c'est à dire quatre images par animation, le personnage principal est le seul à avoir des animations pour toutes les directions, les autres sont animés pour rester sur place et bouger de manière à les rendre plus vivant.
\\
\subsubsection{Les overlays}
Nous avons crée un système d'overlay afin de rendre la map plus dynamique, cette couche de la map affichera l'eau animées et les items de décorations animés comme le bateau.

\subsubsection{Les logo et image de fond}
Enfin, afin de donner une identité visuelle à notre jeu nous avons crée nous-même le logo du jeu , l'icône de la fenêtre et les images de fond du menu. Nous avons également animé ces image à certain endroit dans notre jeu, par exemple le logo de notre jeu est animé dans les menu afin de rendre le jeu plus dynamique ainsi que certains éléments su l'image de fond.
%Ajouter des images

\section{Développement}
\subsection{Logiciel}
\subsection{Méthodes}
\subsection{Les bibliothèques utilisée}
\section{Conception et résultats}
\subsection{Les difficultés rencontrées}
\subsection{Notre expèrience}
\section{Annexes}
\subsection{Les sprites}
\begin{figure}[h]
    \centering
    \includegraphics[height = 2cm]{risette/idle_0.png}
    \includegraphics[height = 2cm]{risette/idle_1.png}
    \includegraphics[height = 2cm]{risette/idle_2.png}
    \includegraphics[height = 2cm]{risette/idle_3.png}
    \\
    \includegraphics[height = 2cm]{haricot/idle_0.png}
    \includegraphics[height = 2cm]{haricot/idle_1.png}
    \includegraphics[height = 2cm]{haricot/idle_2.png}
    \includegraphics[height = 2cm]{haricot/idle_3.png}
    \\
    \includegraphics[height = 2cm]{haricot/back_0.png}
    \includegraphics[height = 2cm]{haricot/back_1.png}
    \includegraphics[height = 2cm]{haricot/back_2.png}
    \includegraphics[height = 2cm]{haricot/back_3.png}
    \\
    \includegraphics[height = 2cm]{haricot/left_0.png}
    \includegraphics[height = 2cm]{haricot/left_1.png}
    \includegraphics[height = 2cm]{haricot/left_2.png}
    \includegraphics[height = 2cm]{haricot/left_3.png}
    \caption{Exemple de sprite du personnage principal}\label{lapin_devant}
\end{figure}



\end{document}



\end{document}
